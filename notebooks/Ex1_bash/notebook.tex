
% Default to the notebook output style

    


% Inherit from the specified cell style.




    
\documentclass[11pt]{article}

    
    
    \usepackage[T1]{fontenc}
    % Nicer default font (+ math font) than Computer Modern for most use cases
    \usepackage{mathpazo}

    % Basic figure setup, for now with no caption control since it's done
    % automatically by Pandoc (which extracts ![](path) syntax from Markdown).
    \usepackage{graphicx}
    % We will generate all images so they have a width \maxwidth. This means
    % that they will get their normal width if they fit onto the page, but
    % are scaled down if they would overflow the margins.
    \makeatletter
    \def\maxwidth{\ifdim\Gin@nat@width>\linewidth\linewidth
    \else\Gin@nat@width\fi}
    \makeatother
    \let\Oldincludegraphics\includegraphics
    % Set max figure width to be 80% of text width, for now hardcoded.
    \renewcommand{\includegraphics}[1]{\Oldincludegraphics[width=.8\maxwidth]{#1}}
    % Ensure that by default, figures have no caption (until we provide a
    % proper Figure object with a Caption API and a way to capture that
    % in the conversion process - todo).
    \usepackage{caption}
    \DeclareCaptionLabelFormat{nolabel}{}
    \captionsetup{labelformat=nolabel}

    \usepackage{adjustbox} % Used to constrain images to a maximum size 
    \usepackage{xcolor} % Allow colors to be defined
    \usepackage{enumerate} % Needed for markdown enumerations to work
    \usepackage{geometry} % Used to adjust the document margins
    \usepackage{amsmath} % Equations
    \usepackage{amssymb} % Equations
    \usepackage{textcomp} % defines textquotesingle
    % Hack from http://tex.stackexchange.com/a/47451/13684:
    \AtBeginDocument{%
        \def\PYZsq{\textquotesingle}% Upright quotes in Pygmentized code
    }
    \usepackage{upquote} % Upright quotes for verbatim code
    \usepackage{eurosym} % defines \euro
    \usepackage[mathletters]{ucs} % Extended unicode (utf-8) support
    \usepackage[utf8x]{inputenc} % Allow utf-8 characters in the tex document
    \usepackage{fancyvrb} % verbatim replacement that allows latex
    \usepackage{grffile} % extends the file name processing of package graphics 
                         % to support a larger range 
    % The hyperref package gives us a pdf with properly built
    % internal navigation ('pdf bookmarks' for the table of contents,
    % internal cross-reference links, web links for URLs, etc.)
    \usepackage{hyperref}
    \usepackage{longtable} % longtable support required by pandoc >1.10
    \usepackage{booktabs}  % table support for pandoc > 1.12.2
    \usepackage[inline]{enumitem} % IRkernel/repr support (it uses the enumerate* environment)
    \usepackage[normalem]{ulem} % ulem is needed to support strikethroughs (\sout)
                                % normalem makes italics be italics, not underlines
    

    
    
    % Colors for the hyperref package
    \definecolor{urlcolor}{rgb}{0,.145,.698}
    \definecolor{linkcolor}{rgb}{.71,0.21,0.01}
    \definecolor{citecolor}{rgb}{.12,.54,.11}

    % ANSI colors
    \definecolor{ansi-black}{HTML}{3E424D}
    \definecolor{ansi-black-intense}{HTML}{282C36}
    \definecolor{ansi-red}{HTML}{E75C58}
    \definecolor{ansi-red-intense}{HTML}{B22B31}
    \definecolor{ansi-green}{HTML}{00A250}
    \definecolor{ansi-green-intense}{HTML}{007427}
    \definecolor{ansi-yellow}{HTML}{DDB62B}
    \definecolor{ansi-yellow-intense}{HTML}{B27D12}
    \definecolor{ansi-blue}{HTML}{208FFB}
    \definecolor{ansi-blue-intense}{HTML}{0065CA}
    \definecolor{ansi-magenta}{HTML}{D160C4}
    \definecolor{ansi-magenta-intense}{HTML}{A03196}
    \definecolor{ansi-cyan}{HTML}{60C6C8}
    \definecolor{ansi-cyan-intense}{HTML}{258F8F}
    \definecolor{ansi-white}{HTML}{C5C1B4}
    \definecolor{ansi-white-intense}{HTML}{A1A6B2}

    % commands and environments needed by pandoc snippets
    % extracted from the output of `pandoc -s`
    \providecommand{\tightlist}{%
      \setlength{\itemsep}{0pt}\setlength{\parskip}{0pt}}
    \DefineVerbatimEnvironment{Highlighting}{Verbatim}{commandchars=\\\{\}}
    % Add ',fontsize=\small' for more characters per line
    \newenvironment{Shaded}{}{}
    \newcommand{\KeywordTok}[1]{\textcolor[rgb]{0.00,0.44,0.13}{\textbf{{#1}}}}
    \newcommand{\DataTypeTok}[1]{\textcolor[rgb]{0.56,0.13,0.00}{{#1}}}
    \newcommand{\DecValTok}[1]{\textcolor[rgb]{0.25,0.63,0.44}{{#1}}}
    \newcommand{\BaseNTok}[1]{\textcolor[rgb]{0.25,0.63,0.44}{{#1}}}
    \newcommand{\FloatTok}[1]{\textcolor[rgb]{0.25,0.63,0.44}{{#1}}}
    \newcommand{\CharTok}[1]{\textcolor[rgb]{0.25,0.44,0.63}{{#1}}}
    \newcommand{\StringTok}[1]{\textcolor[rgb]{0.25,0.44,0.63}{{#1}}}
    \newcommand{\CommentTok}[1]{\textcolor[rgb]{0.38,0.63,0.69}{\textit{{#1}}}}
    \newcommand{\OtherTok}[1]{\textcolor[rgb]{0.00,0.44,0.13}{{#1}}}
    \newcommand{\AlertTok}[1]{\textcolor[rgb]{1.00,0.00,0.00}{\textbf{{#1}}}}
    \newcommand{\FunctionTok}[1]{\textcolor[rgb]{0.02,0.16,0.49}{{#1}}}
    \newcommand{\RegionMarkerTok}[1]{{#1}}
    \newcommand{\ErrorTok}[1]{\textcolor[rgb]{1.00,0.00,0.00}{\textbf{{#1}}}}
    \newcommand{\NormalTok}[1]{{#1}}
    
    % Additional commands for more recent versions of Pandoc
    \newcommand{\ConstantTok}[1]{\textcolor[rgb]{0.53,0.00,0.00}{{#1}}}
    \newcommand{\SpecialCharTok}[1]{\textcolor[rgb]{0.25,0.44,0.63}{{#1}}}
    \newcommand{\VerbatimStringTok}[1]{\textcolor[rgb]{0.25,0.44,0.63}{{#1}}}
    \newcommand{\SpecialStringTok}[1]{\textcolor[rgb]{0.73,0.40,0.53}{{#1}}}
    \newcommand{\ImportTok}[1]{{#1}}
    \newcommand{\DocumentationTok}[1]{\textcolor[rgb]{0.73,0.13,0.13}{\textit{{#1}}}}
    \newcommand{\AnnotationTok}[1]{\textcolor[rgb]{0.38,0.63,0.69}{\textbf{\textit{{#1}}}}}
    \newcommand{\CommentVarTok}[1]{\textcolor[rgb]{0.38,0.63,0.69}{\textbf{\textit{{#1}}}}}
    \newcommand{\VariableTok}[1]{\textcolor[rgb]{0.10,0.09,0.49}{{#1}}}
    \newcommand{\ControlFlowTok}[1]{\textcolor[rgb]{0.00,0.44,0.13}{\textbf{{#1}}}}
    \newcommand{\OperatorTok}[1]{\textcolor[rgb]{0.40,0.40,0.40}{{#1}}}
    \newcommand{\BuiltInTok}[1]{{#1}}
    \newcommand{\ExtensionTok}[1]{{#1}}
    \newcommand{\PreprocessorTok}[1]{\textcolor[rgb]{0.74,0.48,0.00}{{#1}}}
    \newcommand{\AttributeTok}[1]{\textcolor[rgb]{0.49,0.56,0.16}{{#1}}}
    \newcommand{\InformationTok}[1]{\textcolor[rgb]{0.38,0.63,0.69}{\textbf{\textit{{#1}}}}}
    \newcommand{\WarningTok}[1]{\textcolor[rgb]{0.38,0.63,0.69}{\textbf{\textit{{#1}}}}}
    
    
    % Define a nice break command that doesn't care if a line doesn't already
    % exist.
    \def\br{\hspace*{\fill} \\* }
    % Math Jax compatability definitions
    \def\gt{>}
    \def\lt{<}
    % Document parameters
    \title{Basic\_bash}
    
    
    

    % Pygments definitions
    
\makeatletter
\def\PY@reset{\let\PY@it=\relax \let\PY@bf=\relax%
    \let\PY@ul=\relax \let\PY@tc=\relax%
    \let\PY@bc=\relax \let\PY@ff=\relax}
\def\PY@tok#1{\csname PY@tok@#1\endcsname}
\def\PY@toks#1+{\ifx\relax#1\empty\else%
    \PY@tok{#1}\expandafter\PY@toks\fi}
\def\PY@do#1{\PY@bc{\PY@tc{\PY@ul{%
    \PY@it{\PY@bf{\PY@ff{#1}}}}}}}
\def\PY#1#2{\PY@reset\PY@toks#1+\relax+\PY@do{#2}}

\expandafter\def\csname PY@tok@w\endcsname{\def\PY@tc##1{\textcolor[rgb]{0.73,0.73,0.73}{##1}}}
\expandafter\def\csname PY@tok@c\endcsname{\let\PY@it=\textit\def\PY@tc##1{\textcolor[rgb]{0.25,0.50,0.50}{##1}}}
\expandafter\def\csname PY@tok@cp\endcsname{\def\PY@tc##1{\textcolor[rgb]{0.74,0.48,0.00}{##1}}}
\expandafter\def\csname PY@tok@k\endcsname{\let\PY@bf=\textbf\def\PY@tc##1{\textcolor[rgb]{0.00,0.50,0.00}{##1}}}
\expandafter\def\csname PY@tok@kp\endcsname{\def\PY@tc##1{\textcolor[rgb]{0.00,0.50,0.00}{##1}}}
\expandafter\def\csname PY@tok@kt\endcsname{\def\PY@tc##1{\textcolor[rgb]{0.69,0.00,0.25}{##1}}}
\expandafter\def\csname PY@tok@o\endcsname{\def\PY@tc##1{\textcolor[rgb]{0.40,0.40,0.40}{##1}}}
\expandafter\def\csname PY@tok@ow\endcsname{\let\PY@bf=\textbf\def\PY@tc##1{\textcolor[rgb]{0.67,0.13,1.00}{##1}}}
\expandafter\def\csname PY@tok@nb\endcsname{\def\PY@tc##1{\textcolor[rgb]{0.00,0.50,0.00}{##1}}}
\expandafter\def\csname PY@tok@nf\endcsname{\def\PY@tc##1{\textcolor[rgb]{0.00,0.00,1.00}{##1}}}
\expandafter\def\csname PY@tok@nc\endcsname{\let\PY@bf=\textbf\def\PY@tc##1{\textcolor[rgb]{0.00,0.00,1.00}{##1}}}
\expandafter\def\csname PY@tok@nn\endcsname{\let\PY@bf=\textbf\def\PY@tc##1{\textcolor[rgb]{0.00,0.00,1.00}{##1}}}
\expandafter\def\csname PY@tok@ne\endcsname{\let\PY@bf=\textbf\def\PY@tc##1{\textcolor[rgb]{0.82,0.25,0.23}{##1}}}
\expandafter\def\csname PY@tok@nv\endcsname{\def\PY@tc##1{\textcolor[rgb]{0.10,0.09,0.49}{##1}}}
\expandafter\def\csname PY@tok@no\endcsname{\def\PY@tc##1{\textcolor[rgb]{0.53,0.00,0.00}{##1}}}
\expandafter\def\csname PY@tok@nl\endcsname{\def\PY@tc##1{\textcolor[rgb]{0.63,0.63,0.00}{##1}}}
\expandafter\def\csname PY@tok@ni\endcsname{\let\PY@bf=\textbf\def\PY@tc##1{\textcolor[rgb]{0.60,0.60,0.60}{##1}}}
\expandafter\def\csname PY@tok@na\endcsname{\def\PY@tc##1{\textcolor[rgb]{0.49,0.56,0.16}{##1}}}
\expandafter\def\csname PY@tok@nt\endcsname{\let\PY@bf=\textbf\def\PY@tc##1{\textcolor[rgb]{0.00,0.50,0.00}{##1}}}
\expandafter\def\csname PY@tok@nd\endcsname{\def\PY@tc##1{\textcolor[rgb]{0.67,0.13,1.00}{##1}}}
\expandafter\def\csname PY@tok@s\endcsname{\def\PY@tc##1{\textcolor[rgb]{0.73,0.13,0.13}{##1}}}
\expandafter\def\csname PY@tok@sd\endcsname{\let\PY@it=\textit\def\PY@tc##1{\textcolor[rgb]{0.73,0.13,0.13}{##1}}}
\expandafter\def\csname PY@tok@si\endcsname{\let\PY@bf=\textbf\def\PY@tc##1{\textcolor[rgb]{0.73,0.40,0.53}{##1}}}
\expandafter\def\csname PY@tok@se\endcsname{\let\PY@bf=\textbf\def\PY@tc##1{\textcolor[rgb]{0.73,0.40,0.13}{##1}}}
\expandafter\def\csname PY@tok@sr\endcsname{\def\PY@tc##1{\textcolor[rgb]{0.73,0.40,0.53}{##1}}}
\expandafter\def\csname PY@tok@ss\endcsname{\def\PY@tc##1{\textcolor[rgb]{0.10,0.09,0.49}{##1}}}
\expandafter\def\csname PY@tok@sx\endcsname{\def\PY@tc##1{\textcolor[rgb]{0.00,0.50,0.00}{##1}}}
\expandafter\def\csname PY@tok@m\endcsname{\def\PY@tc##1{\textcolor[rgb]{0.40,0.40,0.40}{##1}}}
\expandafter\def\csname PY@tok@gh\endcsname{\let\PY@bf=\textbf\def\PY@tc##1{\textcolor[rgb]{0.00,0.00,0.50}{##1}}}
\expandafter\def\csname PY@tok@gu\endcsname{\let\PY@bf=\textbf\def\PY@tc##1{\textcolor[rgb]{0.50,0.00,0.50}{##1}}}
\expandafter\def\csname PY@tok@gd\endcsname{\def\PY@tc##1{\textcolor[rgb]{0.63,0.00,0.00}{##1}}}
\expandafter\def\csname PY@tok@gi\endcsname{\def\PY@tc##1{\textcolor[rgb]{0.00,0.63,0.00}{##1}}}
\expandafter\def\csname PY@tok@gr\endcsname{\def\PY@tc##1{\textcolor[rgb]{1.00,0.00,0.00}{##1}}}
\expandafter\def\csname PY@tok@ge\endcsname{\let\PY@it=\textit}
\expandafter\def\csname PY@tok@gs\endcsname{\let\PY@bf=\textbf}
\expandafter\def\csname PY@tok@gp\endcsname{\let\PY@bf=\textbf\def\PY@tc##1{\textcolor[rgb]{0.00,0.00,0.50}{##1}}}
\expandafter\def\csname PY@tok@go\endcsname{\def\PY@tc##1{\textcolor[rgb]{0.53,0.53,0.53}{##1}}}
\expandafter\def\csname PY@tok@gt\endcsname{\def\PY@tc##1{\textcolor[rgb]{0.00,0.27,0.87}{##1}}}
\expandafter\def\csname PY@tok@err\endcsname{\def\PY@bc##1{\setlength{\fboxsep}{0pt}\fcolorbox[rgb]{1.00,0.00,0.00}{1,1,1}{\strut ##1}}}
\expandafter\def\csname PY@tok@kc\endcsname{\let\PY@bf=\textbf\def\PY@tc##1{\textcolor[rgb]{0.00,0.50,0.00}{##1}}}
\expandafter\def\csname PY@tok@kd\endcsname{\let\PY@bf=\textbf\def\PY@tc##1{\textcolor[rgb]{0.00,0.50,0.00}{##1}}}
\expandafter\def\csname PY@tok@kn\endcsname{\let\PY@bf=\textbf\def\PY@tc##1{\textcolor[rgb]{0.00,0.50,0.00}{##1}}}
\expandafter\def\csname PY@tok@kr\endcsname{\let\PY@bf=\textbf\def\PY@tc##1{\textcolor[rgb]{0.00,0.50,0.00}{##1}}}
\expandafter\def\csname PY@tok@bp\endcsname{\def\PY@tc##1{\textcolor[rgb]{0.00,0.50,0.00}{##1}}}
\expandafter\def\csname PY@tok@fm\endcsname{\def\PY@tc##1{\textcolor[rgb]{0.00,0.00,1.00}{##1}}}
\expandafter\def\csname PY@tok@vc\endcsname{\def\PY@tc##1{\textcolor[rgb]{0.10,0.09,0.49}{##1}}}
\expandafter\def\csname PY@tok@vg\endcsname{\def\PY@tc##1{\textcolor[rgb]{0.10,0.09,0.49}{##1}}}
\expandafter\def\csname PY@tok@vi\endcsname{\def\PY@tc##1{\textcolor[rgb]{0.10,0.09,0.49}{##1}}}
\expandafter\def\csname PY@tok@vm\endcsname{\def\PY@tc##1{\textcolor[rgb]{0.10,0.09,0.49}{##1}}}
\expandafter\def\csname PY@tok@sa\endcsname{\def\PY@tc##1{\textcolor[rgb]{0.73,0.13,0.13}{##1}}}
\expandafter\def\csname PY@tok@sb\endcsname{\def\PY@tc##1{\textcolor[rgb]{0.73,0.13,0.13}{##1}}}
\expandafter\def\csname PY@tok@sc\endcsname{\def\PY@tc##1{\textcolor[rgb]{0.73,0.13,0.13}{##1}}}
\expandafter\def\csname PY@tok@dl\endcsname{\def\PY@tc##1{\textcolor[rgb]{0.73,0.13,0.13}{##1}}}
\expandafter\def\csname PY@tok@s2\endcsname{\def\PY@tc##1{\textcolor[rgb]{0.73,0.13,0.13}{##1}}}
\expandafter\def\csname PY@tok@sh\endcsname{\def\PY@tc##1{\textcolor[rgb]{0.73,0.13,0.13}{##1}}}
\expandafter\def\csname PY@tok@s1\endcsname{\def\PY@tc##1{\textcolor[rgb]{0.73,0.13,0.13}{##1}}}
\expandafter\def\csname PY@tok@mb\endcsname{\def\PY@tc##1{\textcolor[rgb]{0.40,0.40,0.40}{##1}}}
\expandafter\def\csname PY@tok@mf\endcsname{\def\PY@tc##1{\textcolor[rgb]{0.40,0.40,0.40}{##1}}}
\expandafter\def\csname PY@tok@mh\endcsname{\def\PY@tc##1{\textcolor[rgb]{0.40,0.40,0.40}{##1}}}
\expandafter\def\csname PY@tok@mi\endcsname{\def\PY@tc##1{\textcolor[rgb]{0.40,0.40,0.40}{##1}}}
\expandafter\def\csname PY@tok@il\endcsname{\def\PY@tc##1{\textcolor[rgb]{0.40,0.40,0.40}{##1}}}
\expandafter\def\csname PY@tok@mo\endcsname{\def\PY@tc##1{\textcolor[rgb]{0.40,0.40,0.40}{##1}}}
\expandafter\def\csname PY@tok@ch\endcsname{\let\PY@it=\textit\def\PY@tc##1{\textcolor[rgb]{0.25,0.50,0.50}{##1}}}
\expandafter\def\csname PY@tok@cm\endcsname{\let\PY@it=\textit\def\PY@tc##1{\textcolor[rgb]{0.25,0.50,0.50}{##1}}}
\expandafter\def\csname PY@tok@cpf\endcsname{\let\PY@it=\textit\def\PY@tc##1{\textcolor[rgb]{0.25,0.50,0.50}{##1}}}
\expandafter\def\csname PY@tok@c1\endcsname{\let\PY@it=\textit\def\PY@tc##1{\textcolor[rgb]{0.25,0.50,0.50}{##1}}}
\expandafter\def\csname PY@tok@cs\endcsname{\let\PY@it=\textit\def\PY@tc##1{\textcolor[rgb]{0.25,0.50,0.50}{##1}}}

\def\PYZbs{\char`\\}
\def\PYZus{\char`\_}
\def\PYZob{\char`\{}
\def\PYZcb{\char`\}}
\def\PYZca{\char`\^}
\def\PYZam{\char`\&}
\def\PYZlt{\char`\<}
\def\PYZgt{\char`\>}
\def\PYZsh{\char`\#}
\def\PYZpc{\char`\%}
\def\PYZdl{\char`\$}
\def\PYZhy{\char`\-}
\def\PYZsq{\char`\'}
\def\PYZdq{\char`\"}
\def\PYZti{\char`\~}
% for compatibility with earlier versions
\def\PYZat{@}
\def\PYZlb{[}
\def\PYZrb{]}
\makeatother


    % Exact colors from NB
    \definecolor{incolor}{rgb}{0.0, 0.0, 0.5}
    \definecolor{outcolor}{rgb}{0.545, 0.0, 0.0}



    
    % Prevent overflowing lines due to hard-to-break entities
    \sloppy 
    % Setup hyperref package
    \hypersetup{
      breaklinks=true,  % so long urls are correctly broken across lines
      colorlinks=true,
      urlcolor=urlcolor,
      linkcolor=linkcolor,
      citecolor=citecolor,
      }
    % Slightly bigger margins than the latex defaults
    
    \geometry{verbose,tmargin=1in,bmargin=1in,lmargin=1in,rmargin=1in}
    
    

    \begin{document}
    
    
    \maketitle
    
    

    
    \begin{center}\rule{0.5\linewidth}{\linethickness}\end{center}

\#

Navigating the Unix Shell and Learning Basic Bash

\begin{center}\rule{0.5\linewidth}{\linethickness}\end{center}

\section{Tutorial Outline}\label{tutorial-outline}

\begin{enumerate}
\def\labelenumi{\arabic{enumi}.}
\tightlist
\item
  Getting ready
\item
  How to do it...

  \begin{enumerate}
  \def\labelenumii{\arabic{enumii}.}
  \tightlist
  \item
    Viewing working directory
  \item
    Listing folder contents
  \item
    Changing directories
  \item
    Relative and absolute paths
  \item
    File and folder manipulations
  \item
    Viewing text files
  \item
    grep
  \item
    pipes
  \end{enumerate}
\item
  Basic Bash Scripting
\item
  Bash Variables

  \begin{enumerate}
  \def\labelenumii{\arabic{enumii}.}
  \tightlist
  \item
    Typing: Strings vs. Arithmetic
  \item
    Adding to your PATH
  \item
    Environment Variables and Arguments
  \end{enumerate}
\item
  Conditional Control Flow

  \begin{enumerate}
  \def\labelenumii{\arabic{enumii}.}
  \tightlist
  \item
    If/Elif/Else Statements
  \item
    Comparators
  \item
    For Loops
  \end{enumerate}
\item
  Output to stdout vs. stderr
\item
  Theres more... ***
\end{enumerate}

Learning how to interact with the operating system using a command-line
interface (or terminal) is a required skill in interactive computing and
data analysis. We will use a command-line interface in many of the
exercises through out this class. IPython and the Jupyter Notebook are
typically launched from a terminal. Installing bioinformatics and Python
packages are typically done from a terminal.

In this exercise, we will show the very basics of the Unix shell, which
is natively available in Linux distributions (such as Debian, Ubuntu,
and so on) and macOS. On Windows 10, one can install the \textbf{Windows
Subsystem for Linux}, a command-line interface to a Unix subsystem
integrated with the Windows operating system (see
https://docs.microsoft.com/windows/wsl/about).

    \section{1. Getting ready}\label{getting-ready}

\begin{center}\rule{0.5\linewidth}{\linethickness}\end{center}

Here are the instructions to open a Unix shell on macOS, Linux, and
Windows. bash is the most common Unix shell and this is what we will use
when interacting with the remote server.

On macOS, bring up the Spotlight Search, type \textbf{terminal}, and
press Enter.

On Windows, follow the instructions at
https://docs.microsoft.com/en-us/windows/wsl/install-win10. Then, open
the Windows menu, type \textbf{bash}, and press Enter.

On Linux, open the Dash by clicking on the top-left icon on the desktop,
type \textbf{terminal}, and open the \textbf{Terminal} application.

    \begin{verbatim}
<b>Tip:</b> If you want to run this notebook as bash <b>only</b> in Jupyter, you need to install <b>bash_kernel</b>, available at https://github.com/takluyver/bash_kernel. Open a terminal and type "pip install bash_kernel && python -m bash_kernel.install"
\end{verbatim}

This will install a bash kernel in Jupyter, and it will allow you to run
this recipe's code directly in the Notebook.

    \section{2. How to do it...}\label{how-to-do-it...}

\begin{center}\rule{0.5\linewidth}{\linethickness}\end{center}

The Unix shell comes with hundreds of commands. We will see the most
common ones in tutorial. In each of the code blocks we will be using the
\textbf{\texttt{\%\%bash}} magic command and this will allow us to run
bash code within the jupyter notebook without having to install the bash
kernel. In addition, we can do one line bash commands using the
\textbf{\texttt{!\textless{}cmd\textgreater{}}}, such as
\textbf{\texttt{!pwd}}.

\begin{verbatim}
<b>Note:</b> If we are working directly in the terminal the magic commands should not be used.
\end{verbatim}

    \begin{Verbatim}[commandchars=\\\{\}]
{\color{incolor}In [{\color{incolor}2}]:} \PY{c+c1}{\PYZsh{}lets list all magic commands to make sure we have access to the \PYZpc{}\PYZpc{}bash terminal from jupyter}
        \PY{o}{\PYZpc{}}\PY{k}{lsmagic}
\end{Verbatim}


\begin{Verbatim}[commandchars=\\\{\}]
{\color{outcolor}Out[{\color{outcolor}2}]:} Available line magics:
        \%alias  \%alias\_magic  \%autocall  \%automagic  \%autosave  \%bookmark  \%cat  \%cd  \%clear  \%colors  \%config  \%connect\_info  \%cp  \%debug  \%dhist  \%dirs  \%doctest\_mode  \%ed  \%edit  \%env  \%gui  \%hist  \%history  \%killbgscripts  \%ldir  \%less  \%lf  \%lk  \%ll  \%load  \%load\_ext  \%loadpy  \%logoff  \%logon  \%logstart  \%logstate  \%logstop  \%ls  \%lsmagic  \%lx  \%macro  \%magic  \%man  \%matplotlib  \%mkdir  \%more  \%mv  \%notebook  \%page  \%pastebin  \%pdb  \%pdef  \%pdoc  \%pfile  \%pinfo  \%pinfo2  \%popd  \%pprint  \%precision  \%profile  \%prun  \%psearch  \%psource  \%pushd  \%pwd  \%pycat  \%pylab  \%qtconsole  \%quickref  \%recall  \%rehashx  \%reload\_ext  \%rep  \%rerun  \%reset  \%reset\_selective  \%rm  \%rmdir  \%run  \%save  \%sc  \%set\_env  \%store  \%sx  \%system  \%tb  \%time  \%timeit  \%unalias  \%unload\_ext  \%who  \%who\_ls  \%whos  \%xdel  \%xmode
        
        Available cell magics:
        \%\%!  \%\%HTML  \%\%SVG  \%\%bash  \%\%capture  \%\%debug  \%\%file  \%\%html  \%\%javascript  \%\%js  \%\%latex  \%\%markdown  \%\%perl  \%\%prun  \%\%pypy  \%\%python  \%\%python2  \%\%python3  \%\%ruby  \%\%script  \%\%sh  \%\%svg  \%\%sx  \%\%system  \%\%time  \%\%timeit  \%\%writefile
        
        Automagic is ON, \% prefix IS NOT needed for line magics.
\end{Verbatim}
            
    \textbf{A.} The terminal lets us write text commands with the keyboard.
We execute them by pressing Enter, and the output is displayed below the
command. The working directory is the directory of our file system that
is currently "active" in the terminal. We can get the absolute path of
the working directory as follows:

    \begin{Verbatim}[commandchars=\\\{\}]
{\color{incolor}In [{\color{incolor}3}]:} \PYZpc{}\PYZpc{}bash
        
        \PY{n+nb}{pwd}
\end{Verbatim}


    \begin{Verbatim}[commandchars=\\\{\}]
/home/WWU/biology/GenomeAnalysis/notebooks/Ex1\_bash

    \end{Verbatim}

    or

    \begin{Verbatim}[commandchars=\\\{\}]
{\color{incolor}In [{\color{incolor}4}]:} \PY{o}{!}pwd
\end{Verbatim}


    \begin{Verbatim}[commandchars=\\\{\}]
/home/WWU/biology/GenomeAnalysis/notebooks/Ex1\_bash

    \end{Verbatim}

    \textbf{B.} We can use the \textbf{\texttt{ls}} command to list all
files and subdirectories in the working directory as follows:

    \begin{Verbatim}[commandchars=\\\{\}]
{\color{incolor}In [{\color{incolor}7}]:} \PYZpc{}\PYZpc{}bash
        
        ls \PYZti{}/GenomeAnalysis/
\end{Verbatim}


    \begin{Verbatim}[commandchars=\\\{\}]
data
notebooks

    \end{Verbatim}

    The \textbf{\texttt{-l}} option displays the directory contents as a
detailed list, showing the permissions and owner of the files, the file
sizes, and the last modified dates. Most shell commands come with many
options that alter their behavior and that can be arbitrarily combined.

    \begin{Verbatim}[commandchars=\\\{\}]
{\color{incolor}In [{\color{incolor}8}]:} \PYZpc{}\PYZpc{}bash
        
        ls \PYZhy{}l \PYZti{}/GenomeAnalysis/
\end{Verbatim}


    \begin{Verbatim}[commandchars=\\\{\}]
total 8
drwxrwxr-x 3 zinkgrm domain users 4096 Aug 30  2018 data
drwxrwxr-x 8 zinkgrm domain users 4096 Mar  4 15:13 notebooks

    \end{Verbatim}

    \textbf{C.} We use the \textbf{\texttt{cd}} command to navigate between
subdirectories. The current directory is named \textbf{\texttt{.}}
(single dot), and the parent directory is named \textbf{\texttt{../}}
(double dot):

    \begin{Verbatim}[commandchars=\\\{\}]
{\color{incolor}In [{\color{incolor}9}]:} \PYZpc{}\PYZpc{}bash
        
        \PY{n+nb}{cd} data
        
        \PY{n+nb}{pwd}
\end{Verbatim}


    \begin{Verbatim}[commandchars=\\\{\}]
/home/WWU/biology/GenomeAnalysis/notebooks/Ex1\_bash/data

    \end{Verbatim}

    \begin{Verbatim}[commandchars=\\\{\}]
{\color{incolor}In [{\color{incolor}10}]:} \PYZpc{}\PYZpc{}bash
         
         \PY{n+nb}{cd} ../
         
         \PY{n+nb}{pwd}
\end{Verbatim}


    \begin{Verbatim}[commandchars=\\\{\}]
/home/WWU/biology/GenomeAnalysis/notebooks

    \end{Verbatim}

    \textbf{D.} Paths can be specified as relative (depending on a reference
directory, generally the working directory) or absolute. The home
directory, specified as \textbf{\texttt{\textasciitilde{}}}, contains
the user's personal files. Configuration files are often stored in a
directory like \textbf{\texttt{\textasciitilde{}/.program\_name}}. For
example, \textbf{\texttt{\textasciitilde{}/.ipython}} contains
configuration files of IPython:

    \begin{Verbatim}[commandchars=\\\{\}]
{\color{incolor}In [{\color{incolor}14}]:} \PYZpc{}\PYZpc{}bash
         
         ls \PYZhy{}la \PYZti{}/.ipython
\end{Verbatim}


    \begin{Verbatim}[commandchars=\\\{\}]
total 20
drwxr-xr-x  5 zinkgrm domain users 4096 Aug 30  2018 .
drwxr-xr-x 18 zinkgrm domain users 4096 Mar  1 11:56 ..
drwxr-xr-x  2 zinkgrm domain users 4096 Aug 30  2018 extensions
drwxr-xr-x  2 zinkgrm domain users 4096 Aug 30  2018 nbextensions
drwxr-xr-x  7 zinkgrm domain users 4096 Mar  1 16:24 profile\_default

    \end{Verbatim}

    In most terminals, we can use the arrow keys on the keyboard to navigate
in the history of past commands. Also, the Tab key enables tab
completion, which automatically completes the first characters of a
command or a file. For example, typing
\textbf{\texttt{ls\ -la\ \textasciitilde{}/.ipy}} and pressing Tab would
automatically complete to
\textbf{\texttt{ls\ -la\ \textasciitilde{}/.ipython}}, or it would
present the list of possible options if there are several files or
directories that begin with \textbf{\texttt{\textasciitilde{}/.ipy}}.

    \textbf{E.} We can create, move, rename, copy, delete files and
directories from the terminal:

    \begin{Verbatim}[commandchars=\\\{\}]
{\color{incolor}In [{\color{incolor}15}]:} \PYZpc{}\PYZpc{}bash
         
         ls
\end{Verbatim}


    \begin{Verbatim}[commandchars=\\\{\}]
Basic\_bash.ipynb
basic.txt
data
HelloUser.sh
helloworld.sh
large.fasta
medium.fasta
printargs.sh
small.fasta

    \end{Verbatim}

    \begin{Verbatim}[commandchars=\\\{\}]
{\color{incolor}In [{\color{incolor}30}]:} \PYZpc{}\PYZpc{}bash
         
         \PY{c+c1}{\PYZsh{}\PYZsh{} update to new content \PYZsh{}\PYZsh{}}
         \PY{c+c1}{\PYZsh{} We create an empty directory:}
         mkdir \PYZti{}/nb\PYZus{}files
         
         \PY{c+c1}{\PYZsh{} we check to see if it was made}
         ls
\end{Verbatim}


    \begin{Verbatim}[commandchars=\\\{\}]
Basic\_bash.ipynb
basic.txt
data
HelloUser.sh
helloworld.sh
large.fasta
medium.fasta
nb\_files
printargs.sh
small.fasta

    \end{Verbatim}

    \begin{Verbatim}[commandchars=\\\{\}]
{\color{incolor}In [{\color{incolor}31}]:} \PYZpc{}\PYZpc{}bash
         
         \PY{c+c1}{\PYZsh{} We copy all notebook files into the new directory:}
         cp *.ipynb \PYZti{}/nb\PYZus{}files
         
         \PY{c+c1}{\PYZsh{} We rename the directory and check to see if it was renamed:}
         mv \PYZti{}/nb\PYZus{}files \PYZti{}/notebook\PYZus{}files
         ls
         
         \PY{c+c1}{\PYZsh{} We check what files are contained in the new directory:}
         ls \PYZti{}/notebook\PYZus{}files
\end{Verbatim}


    \begin{Verbatim}[commandchars=\\\{\}]
Basic\_bash.ipynb
basic.txt
data
HelloUser.sh
helloworld.sh
large.fasta
medium.fasta
notebook\_files
printargs.sh
small.fasta
Basic\_bash.ipynb

    \end{Verbatim}

    \begin{Verbatim}[commandchars=\\\{\}]
{\color{incolor}In [{\color{incolor}32}]:} \PYZpc{}\PYZpc{}bash
         
         \PY{c+c1}{\PYZsh{}This command will fail because the folder is not empty}
         rmdir \PYZti{}/notebook\PYZus{}files
         
         \PY{c+c1}{\PYZsh{} First remove files in the folder}
         rm \PYZti{}/notebook\PYZus{}files/*
         
         \PY{c+c1}{\PYZsh{} Then remove folder}
         rmdir \PYZti{}/notebook\PYZus{}files
         
         \PY{c+c1}{\PYZsh{} Check}
         ls \PYZti{}/
\end{Verbatim}


    \begin{Verbatim}[commandchars=\\\{\}]
Basic\_bash.ipynb
basic.txt
data
HelloUser.sh
helloworld.sh
large.fasta
medium.fasta
printargs.sh
small.fasta

    \end{Verbatim}

    \begin{Verbatim}[commandchars=\\\{\}]
rmdir: failed to remove 'notebook\_files': Directory not empty

    \end{Verbatim}

    The \textbf{\texttt{rm}} command lets us delete files and directories.
The \textbf{\texttt{rm\ -rf}} path deletes the given path recursively,
even if subdirectories are not empty. It is an extremely dangerous
command as it cannot be undone: the files are immediately and
permanently deleted, they do not go into a trash directory first. See
https://github.com/sindresorhus/guides/blob/master/how-not-to-rm-yourself.md
for more details.

    \textbf{F.} There are several useful commands to deal with text files:

    \begin{Verbatim}[commandchars=\\\{\}]
{\color{incolor}In [{\color{incolor}38}]:} \PYZpc{}\PYZpc{}bash
         
         \PY{c+c1}{\PYZsh{} Show the first three lines of a text file:}
         head \PYZhy{}n \PY{l+m}{3} basic.txt
\end{Verbatim}


    \begin{Verbatim}[commandchars=\\\{\}]
Hi 
this 
is test file 

    \end{Verbatim}

    \begin{Verbatim}[commandchars=\\\{\}]
{\color{incolor}In [{\color{incolor}39}]:} \PYZpc{}\PYZpc{}bash
         
         \PY{c+c1}{\PYZsh{} Show the last line of a text file:}
         tail \PYZhy{}n \PY{l+m}{1} basic.txt
\end{Verbatim}


    \begin{Verbatim}[commandchars=\\\{\}]
ABCD EFG
    \end{Verbatim}

    \begin{Verbatim}[commandchars=\\\{\}]
{\color{incolor}In [{\color{incolor}36}]:} \PYZpc{}\PYZpc{}bash
         
         \PY{c+c1}{\PYZsh{} We display some text:}
         \PY{n+nb}{echo} \PY{l+s+s2}{\PYZdq{}Hello world!\PYZdq{}}
\end{Verbatim}


    \begin{Verbatim}[commandchars=\\\{\}]
Hello world!

    \end{Verbatim}

    \begin{Verbatim}[commandchars=\\\{\}]
{\color{incolor}In [{\color{incolor}37}]:} \PYZpc{}\PYZpc{}bash
         
         \PY{c+c1}{\PYZsh{} We redirect the output of a command to}
         \PY{c+c1}{\PYZsh{} a text file with `\PYZgt{}`:}
         \PY{n+nb}{echo} \PY{l+s+s2}{\PYZdq{}Hello world!\PYZdq{}} \PYZgt{} myfile.txt
         \PY{c+c1}{\PYZsh{} We display the entire contents of the file:}
         cat myfile.txt
\end{Verbatim}


    \begin{Verbatim}[commandchars=\\\{\}]
Hello world!

    \end{Verbatim}

    \textbf{G.} The grep command lets us search substrings in text and is an
acronym for ``Global Regular Expression Print''. Now let's go on to some
practical examples of using grep. To better understand the results, I've
created a simple text file \textbf{\texttt{basic.txt}} on which we will
run our grep searches; the file contains the following lines:

    \begin{Verbatim}[commandchars=\\\{\}]
{\color{incolor}In [{\color{incolor}18}]:} \PY{o}{!}head basic.txt
\end{Verbatim}


    \begin{Verbatim}[commandchars=\\\{\}]
Hi 
this 
is test file 
to carry out few regular expressions 
practical with grep 
123 456 
Abcd
ABCD EFG
    \end{Verbatim}

    We can do a case-insensitive search using the \textbf{\texttt{-i}}
option:

    \begin{Verbatim}[commandchars=\\\{\}]
{\color{incolor}In [{\color{incolor}40}]:} \PYZpc{}\PYZpc{}bash
         
         grep \PYZhy{}i \PY{l+s+s1}{\PYZsq{}abcd\PYZsq{}} basic.txt
\end{Verbatim}


    \begin{Verbatim}[commandchars=\\\{\}]
Abcd
ABCD EFG

    \end{Verbatim}

    \begin{verbatim}
<b>Note:</b> By default grep prints the results of the entire line. To restrict the print to the search pattern, add the __`-o`__ option.
\end{verbatim}

We can count the number of occurances of a match by using the
\textbf{\texttt{-c}} option:

    \begin{Verbatim}[commandchars=\\\{\}]
{\color{incolor}In [{\color{incolor}41}]:} \PYZpc{}\PYZpc{}bash
         
         grep \PYZhy{}c \PY{l+s+s1}{\PYZsq{}is\PYZsq{}} basic.txt
\end{Verbatim}


    \begin{Verbatim}[commandchars=\\\{\}]
2

    \end{Verbatim}

    We can use the \textbf{\texttt{-Eo}} option to find all instances that
match, test followed by a word (using regular expressions):

    \begin{Verbatim}[commandchars=\\\{\}]
{\color{incolor}In [{\color{incolor}42}]:} \PYZpc{}\PYZpc{}bash
         
         grep \PYZhy{}Eo \PY{l+s+s2}{\PYZdq{}test \PYZbs{}w+\PYZdq{}} basic.txt
\end{Verbatim}


    \begin{Verbatim}[commandchars=\\\{\}]
test file

    \end{Verbatim}

    \textbf{H.} A major strength of the Unix shell is that commands can be
combined with pipes: the output of one command can be directly
transferred to the input of another command:

    \begin{Verbatim}[commandchars=\\\{\}]
{\color{incolor}In [{\color{incolor}43}]:} \PYZpc{}\PYZpc{}bash
         
         \PY{n+nb}{echo} \PY{l+s+s2}{\PYZdq{}This is a Unix shell\PYZdq{}} \PY{p}{|} grep \PYZhy{}Eo \PY{l+s+s2}{\PYZdq{}Unix \PYZbs{}w+\PYZdq{}}
\end{Verbatim}


    \begin{Verbatim}[commandchars=\\\{\}]
Unix shell

    \end{Verbatim}

    \subsection{3. Basic Bash Scripting}\label{basic-bash-scripting}

What we have been doing so far is single operations per command line.
Next, we are going to see how to combine multiple operations and some
programming logic to build small scripts to perform basic utilities.
https://www.usna.edu/Users/cs/aviv/classes/ic221/s17/lab/02/lab.html

    \subsubsection{Hello World}\label{hello-world}

    Scripting is synonymous with programming --- it is programming ---
except it's generally considered more light weight. For the scripts
we'll write in this class, the commands will be stored in a file, but
that file is interpreted by the bash shell. That means each line
actually runs like a sequence of command lines executed in the order
they apear in the file as if they were typed onto the command line.

You must indicate that you wish for the your script to be treated as
such by designating which program should be interpreting the command
lines. To make that designation, we use a shabang symbol
\textbf{\texttt{\#!}} symbol followed by the path to a program. For this
section, that program is going to be bash, namely the shell. To find out
the path we can use the command \textbf{\texttt{which\ bash}}.

Next, we want to write our first program using a text editor
(\textbf{\texttt{nano}}), namely to print "Hello World," so let's take a
look at that first.

    \begin{Verbatim}[commandchars=\\\{\}]
{\color{incolor}In [{\color{incolor}47}]:} \PYZpc{}\PYZpc{}bash
         
         less helloworld.sh
\end{Verbatim}


    \begin{Verbatim}[commandchars=\\\{\}]
\#!/bin/bash

echo "Hello World"  \# This is a comment


    \end{Verbatim}

    You can find the script helloworld.sh in the examples folder in the lab
directory. By convention, all bash scripts, or shell scripts, have a
file type \textbf{.sh}, and we will use that convention in this class.

On line 1, there is the shebang tells the shell what programming
language to use to interpret the code, such as bash.

On line 3, there is the actual command to print to standard out, "Hello
World", as well as a comment, anything following a "\#". The echo
command echoes back anything it is given to stdout, in this case, the
string "Hello World."

    Several command-line text editors are available, such as
\textbf{\texttt{pico}}, \textbf{\texttt{nano}}, or \textbf{\texttt{vi}}.
Learning these text editors requires time and effort, especially vi. One
of the most user friendly text editors is \textbf{\texttt{nano}}. From a
terminal, create a script in your home directory
\textbf{\texttt{nano\ \textasciitilde{}/helloworld.sh}}. We will go
through a tutorial about how to use nano and and the associated
keystrokes to navigate. To open a terminal, go the the Jupyter home
screen and select New -\textgreater{} Terminal. This will open a new
page with a bash cammand line terminal.

Next, you have to make the file executable using \textbf{\texttt{chmod}}
before we can execute the program. You will notice in the
\textbf{\texttt{ls\ -l}} output that the file now has x associated with
the permissions.

    \begin{Verbatim}[commandchars=\\\{\}]
{\color{incolor}In [{\color{incolor} }]:} \PYZpc{}\PYZpc{}bash 
        
        chmod +x \PYZti{}/helloworld.sh
        
        ls \PYZhy{}l \PYZti{}/helloworld.sh \PY{c+c1}{\PYZsh{}\PYZsh{} Make sure this file shows up in your home directory or you will get an error}
\end{Verbatim}


    Now you are ready to execute with
\textbf{\texttt{\textasciitilde{}/helloworld.sh}} or
\textbf{\texttt{bash\ \textasciitilde{}/helloworld.sh}}

    \begin{Verbatim}[commandchars=\\\{\}]
{\color{incolor}In [{\color{incolor}6}]:} \PYZpc{}\PYZpc{}bash
        
        \PYZti{}/helloworld.sh
\end{Verbatim}


    \begin{Verbatim}[commandchars=\\\{\}]
Hello World

    \end{Verbatim}

    \subsection{4. Bash Variables}\label{bash-variables}

You can assign variables in bash just like in other programming
languages, but you refer to variables using a \$ symbol. Note that
spacing is important, the assignment must occur immediately following
the equal sign. If there are spaces in the string value, use quotes.

    \begin{Verbatim}[commandchars=\\\{\}]
{\color{incolor}In [{\color{incolor}27}]:} \PYZpc{}\PYZpc{}bash
         
         \PY{n+nv}{name}\PY{o}{=}matt
         \PY{n+nb}{echo} \PY{l+s+s2}{\PYZdq{}}\PY{l+s+s2}{Your name is }\PY{n+nv}{\PYZdl{}name}\PY{l+s+s2}{\PYZdq{}}
\end{Verbatim}


    \begin{Verbatim}[commandchars=\\\{\}]
Your name is matt

    \end{Verbatim}

    Information can also be captured from the stdout of a program using
\$(...). The below example uses \textbf{\texttt{wc}} with the
\textbf{\texttt{-l}} option to count the number of lines in the file.

    \begin{Verbatim}[commandchars=\\\{\}]
{\color{incolor}In [{\color{incolor}35}]:} \PYZpc{}\PYZpc{}bash
         
         \PY{n+nv}{count}\PY{o}{=}\PY{k}{\PYZdl{}(}wc \PYZhy{}l \PYZlt{} large.fasta\PY{k}{)}
         
         \PY{n+nb}{echo} \PY{n+nv}{\PYZdl{}count}
\end{Verbatim}


    \begin{Verbatim}[commandchars=\\\{\}]
111259

    \end{Verbatim}

    \begin{Verbatim}[commandchars=\\\{\}]
{\color{incolor}In [{\color{incolor}11}]:} \PYZpc{}\PYZpc{}bash
         
         \PY{n+nv}{name}\PY{o}{=}\PY{l+s+s2}{\PYZdq{}Matt Zink\PYZdq{}}
         \PY{n+nb}{echo} \PY{l+s+s2}{\PYZdq{}}\PY{l+s+s2}{Your name is }\PY{n+nv}{\PYZdl{}name}\PY{l+s+s2}{\PYZdq{}}
\end{Verbatim}


    \begin{Verbatim}[commandchars=\\\{\}]
Your name is Matt Zink

    \end{Verbatim}

    Finally, variable replacement, that is substituting a variable for its
value, occurs throughout a script, including in quotes, like above.

    \subsection{A. Typing: Strings vs.
Arithmetic}\label{a.-typing-strings-vs.-arithmetic}

One major difference between bash, which is a scripting language, and R
or python, is that it is not strongly typed. That means, you don't need
to identify the type of the data you're storing to a variable. Instead,
you have to provide some context for how you want the data to be
interpreted, otherwise, by default, it will be treated like strings and
+ is concatenation.

    \begin{Verbatim}[commandchars=\\\{\}]
{\color{incolor}In [{\color{incolor}49}]:} \PYZpc{}\PYZpc{}bash
         
         \PY{n+nv}{n}\PY{o}{=}\PY{l+m}{1}
         \PY{n+nv}{n}\PY{o}{=}\PY{l+m}{1}+\PY{n+nv}{\PYZdl{}n}
         \PY{n+nb}{echo} \PY{n+nv}{\PYZdl{}n}   \PY{c+c1}{\PYZsh{}\PYZlt{}\PYZhy{}\PYZhy{} print \PYZdq{}1+1\PYZdq{} not 2!}
\end{Verbatim}


    \begin{Verbatim}[commandchars=\\\{\}]
1+1

    \end{Verbatim}

    To peform arithmetic operations, you use the \textbf{\texttt{let}}
operations.

    \begin{Verbatim}[commandchars=\\\{\}]
{\color{incolor}In [{\color{incolor}50}]:} \PYZpc{}\PYZpc{}bash
         
         \PY{n+nv}{n}\PY{o}{=}\PY{l+m}{1}
         \PY{n+nb}{let} \PY{n+nv}{n}\PY{o}{=}\PY{l+m}{1}+\PY{n+nv}{\PYZdl{}n} \PY{c+c1}{\PYZsh{}\PYZlt{}\PYZhy{}\PYZhy{} use quotes if you need white space}
         \PY{n+nb}{echo} \PY{n+nv}{\PYZdl{}n}   \PY{c+c1}{\PYZsh{}\PYZlt{}\PYZhy{}\PYZhy{} prints 2!}
\end{Verbatim}


    \begin{Verbatim}[commandchars=\\\{\}]
2

    \end{Verbatim}

    There is also no such things as floats in bash. Everything is a numeric
integer.

    \subsection{B. Adding to your PATH}\label{b.-adding-to-your-path}

As you develop new tools for this class, you would like to be able to
call on them directly, like you do other commmand line tools. To do
this, those scripts/programs need to exist on the \$PATH.

    \begin{Verbatim}[commandchars=\\\{\}]
{\color{incolor}In [{\color{incolor}52}]:} \PYZpc{}\PYZpc{}bash
         
         \PY{c+c1}{\PYZsh{} see what \PYZdl{}HOME and \PYZdl{}PATH are set to}
         \PY{n+nb}{echo} \PY{n+nv}{\PYZdl{}HOME}
         \PY{n+nb}{echo} \PY{n+nv}{\PYZdl{}PATH}
\end{Verbatim}


    \begin{Verbatim}[commandchars=\\\{\}]
/home/WWU/pollard
/bin:/usr/local/sbin:/usr/local/bin:/usr/sbin:/usr/bin:/opt/anaconda3/bin

    \end{Verbatim}

    The standard, Unix way to do this, is to create a directory in your home
folder called \textbf{\texttt{\textasciitilde{}/bin}} and place all
newly create commands there. Unfortunately, your bin directory may not
be on the search path yet. To add it, you have to update the enviroment
variable PATH. The PATH can be permanently updated by modifying the PATH
line in the file called \texttt{.bashrc} using a text editor, such as
\textbf{\texttt{nano}}.

    \begin{verbatim}
<b>Note:</b> When specifying a PATH, the folders are seperated by _`:`_.
\end{verbatim}

    \begin{Verbatim}[commandchars=\\\{\}]
{\color{incolor}In [{\color{incolor} }]:} \PYZpc{}\PYZpc{}bash
        
        mkdir \PYZti{}/bin  \PY{c+c1}{\PYZsh{}create bin folder}
        mv \PYZti{}/helloworld.sh \PYZti{}/bin/   \PY{c+c1}{\PYZsh{}mv script to the new folder}
        
        \PY{n+nb}{export} \PY{n+nv}{PATH}\PY{o}{=}\PY{n+nv}{\PYZdl{}HOME}/bin:\PY{n+nv}{\PYZdl{}PATH}   \PY{c+c1}{\PYZsh{}modify your path to include the \PYZdl{}HOME/bin:}
        \PY{n+nb}{echo} \PY{n+nv}{\PYZdl{}PATH}
\end{Verbatim}


    This will place \textbf{\texttt{\textasciitilde{}/bin}} at the start of
the search path while preserving prior values on the path. The export
command sets the assignment globally, across all scripts and programs
that are to be run from the shell.

    \subsection{C. Environment Variables and
Arguments}\label{c.-environment-variables-and-arguments}

In a bash script, and in bash shells in general, there are a number of
environment variables available for convenience. Once such environment
variable we've seen previously is the PATH variable:

By convention all environment variables are upper case. Below are some
really useful ones to know:

\begin{longtable}[c]{@{}ll@{}}
\toprule
Variable & Description\tabularnewline
\midrule
\endhead
\$USER & the current user\tabularnewline
\$HOME & the home directory\tabularnewline
\$PWD & The current working directory\tabularnewline
\$SHELL & The name of the shell you are currently running\tabularnewline
\bottomrule
\end{longtable}

    \begin{Verbatim}[commandchars=\\\{\}]
{\color{incolor}In [{\color{incolor}55}]:} \PYZpc{}\PYZpc{}bash
         
         \PY{n+nb}{echo} \PY{n+nv}{\PYZdl{}PWD}
\end{Verbatim}


    \begin{Verbatim}[commandchars=\\\{\}]
/home/WWU/biology/GenomeAnalysis/notebooks/Ex1\_bash

    \end{Verbatim}

    For example, this program \textbf{\texttt{HelloUser.sh}} prints a
helpful message

    \begin{Verbatim}[commandchars=\\\{\}]
{\color{incolor}In [{\color{incolor}56}]:} \PYZpc{}\PYZpc{}bash
         
         \PY{c+c1}{\PYZsh{}look at the script}
         less HelloUser.sh
\end{Verbatim}


    \begin{Verbatim}[commandchars=\\\{\}]
\#!/bin/bash

echo "Hello \$USER!"
echo "You are currently here: \$PWD"
echo "But, your home directory here: \$HOME"

    \end{Verbatim}

    \begin{Verbatim}[commandchars=\\\{\}]
{\color{incolor}In [{\color{incolor}57}]:} \PYZpc{}\PYZpc{}bash
         
         \PY{c+c1}{\PYZsh{}execute the script}
         bash HelloUser.sh
\end{Verbatim}


    \begin{Verbatim}[commandchars=\\\{\}]
Hello pollard!
You are currently here: /home/WWU/biology/GenomeAnalysis/notebooks/Ex1\_bash
But, your home directory here: /home/WWU/pollard

    \end{Verbatim}

    Additionally, there are environment variables for arguments passed to
the shell script. These are \texttt{\$0}, \texttt{\$1}, \texttt{\$2},
etc, where the number refers to the command line argument. For example,
consider the script, \textbf{\texttt{printargs.sh}}

    \begin{Verbatim}[commandchars=\\\{\}]
{\color{incolor}In [{\color{incolor}58}]:} \PYZpc{}\PYZpc{}bash
         
         less printargs.sh
\end{Verbatim}


    \begin{Verbatim}[commandchars=\\\{\}]
\#!/bin/bash                                                                                                        

echo "arg 0: \$0"
echo "arg 1: \$1"
echo "arg 2: \$2"
echo "arg 3: \$3"
echo "arg 4: \$4"


    \end{Verbatim}

    \begin{Verbatim}[commandchars=\\\{\}]
{\color{incolor}In [{\color{incolor}59}]:} \PYZpc{}\PYZpc{}bash
         
         \PY{c+c1}{\PYZsh{}args   0         1 2 3 4}
         \PY{c+c1}{\PYZsh{}       |         | | | |}
         bash printargs.sh x y z
\end{Verbatim}


    \begin{Verbatim}[commandchars=\\\{\}]
arg 0: printargs.sh
arg 1: x
arg 2: y
arg 3: z
arg 4: 

    \end{Verbatim}

    Argument 0 always refers to the script, and each \texttt{\$1} refers to
the arguments to the script. The script prints the first 4 arguments,
but there isn't a fourth argument. \texttt{\$4} is treated as the empty
string.

    To refer to all arguments, not including \texttt{\$0}, use \texttt{\$*}.
\texttt{\$\#} is special variable set to the number of arguments. Try
adding the following code to the \textbf{\texttt{printargs.sh}} script.

    \begin{Verbatim}[commandchars=\\\{\}]
{\color{incolor}In [{\color{incolor}61}]:} \PYZpc{}\PYZpc{}bash
         
         \PY{n+nb}{echo} \PY{l+s+s2}{\PYZdq{}}\PY{l+s+s2}{There are }\PY{n+nv}{\PYZdl{}\PYZsh{}}\PY{l+s+s2}{ number of args}\PY{l+s+s2}{\PYZdq{}}
         \PY{n+nb}{echo} \PY{l+s+s2}{\PYZdq{}}\PY{l+s+s2}{Arguments \PYZus{}not\PYZus{} including the name of the script: }\PY{n+nv}{\PYZdl{}*}\PY{l+s+s2}{\PYZdq{}}
\end{Verbatim}


    \begin{Verbatim}[commandchars=\\\{\}]
There are 0 number of args
Arguments \_not\_ including the name of the script: 

    \end{Verbatim}

    \subsection{5. Conditional Control Flow}\label{conditional-control-flow}

Like any reasonable programming environment, you want to do more than
just print things out in an iterative fashion. We need a mechanism to
change the program based on some condition, like if/else statements and
looping.

\subsubsection{A. If/Elif/Else
Statements}\label{a.-ifelifelse-statements}

The format of an if statement is like so:

    \begin{Shaded}
\begin{Highlighting}[]
\KeywordTok{if} \KeywordTok{cmd}
\KeywordTok{then} 
    \KeywordTok{cmd}
\KeywordTok{elif} \KeywordTok{cmd}
\KeywordTok{then}
    \KeywordTok{cmd}
\KeywordTok{else}
    \KeywordTok{cmd}  
\KeywordTok{fi}        \CommentTok{#<-- close off the if statemet with fi}
\end{Highlighting}
\end{Shaded}

    The \texttt{if\ cmd} part proceeds if the cmd succeeds, that is, it
doesn't have exit with failure. In general, the cmd used in \texttt{if}
blocks and others are the {[} {]} which checks conditions. With the {[}
{]} you can do simple comparators and other operations. Here is a an
examples script that checks if the script was run from the home
directory:

    \begin{Verbatim}[commandchars=\\\{\}]
{\color{incolor}In [{\color{incolor}62}]:} \PYZpc{}\PYZpc{}bash
         
         \PY{n+nb}{echo} \PY{l+s+s2}{\PYZdq{}}\PY{l+s+s2}{Hello }\PY{n+nv}{\PYZdl{}USER}\PY{l+s+s2}{\PYZdq{}}
         
         \PY{k}{if} \PY{o}{[} \PY{n+nv}{\PYZdl{}HOME} \PY{o}{=}\PY{o}{=} \PY{n+nv}{\PYZdl{}PWD} \PY{o}{]}
         \PY{k}{then}
             \PY{n+nb}{echo} \PY{l+s+s2}{\PYZdq{}}\PY{l+s+s2}{Good, you\PYZsq{}re in your home directory: }\PY{n+nv}{\PYZdl{}HOME}\PY{l+s+s2}{\PYZdq{}}
         \PY{k}{else}
             \PY{n+nb}{echo} \PY{l+s+s2}{\PYZdq{}What are you doing away from home?!?\PYZdq{}}
         
             \PY{n+nb}{cd} \PY{n+nv}{\PYZdl{}HOME}
         
             \PY{n+nb}{echo} \PY{l+s+s2}{\PYZdq{}}\PY{l+s+s2}{Now you are in your home directory: }\PY{n+nv}{\PYZdl{}PWD}\PY{l+s+s2}{\PYZdq{}}
         \PY{k}{fi}
\end{Verbatim}


    \begin{Verbatim}[commandchars=\\\{\}]
Hello pollard
What are you doing away from home?!?
Now you are in your home directory: /home/WWU/pollard

    \end{Verbatim}

    There are two things of note here. First, the
\texttt{{[}\ \$HOME\ ==\ \$PWD\ {]}} is a logical command that succeeds
only when \texttt{\$HOME} is the present working directory. Also, when
you change directories within a script, you are not changing the
directory of the shell you ran the script from, just the present working
directory of the script itself.

    \subsubsection{B. Comparators}\label{b.-comparators}

Conditions in bash exist within brackets, {[} {]}, but the {[} {]} is
actually a command that returns true if the conditions were met. For
string comparators, you can use \texttt{=} and \texttt{!} but for
numerics, you need to use options to the {[} {]} command. See below for
the varied usages.

    \begin{verbatim}
<b>Note:</b> Remember when we create variables in bash, we create then by __`var=value`__ and when the variable is recalled or used, we use __`$var`__.
\end{verbatim}

    \begin{Verbatim}[commandchars=\\\{\}]
{\color{incolor}In [{\color{incolor}1}]:} \PYZpc{}\PYZpc{}bash
        
        \PY{c+c1}{\PYZsh{}string                                                                                                            }
        \PY{n+nv}{var}\PY{o}{=}adam
        \PY{n+nv}{user}\PY{o}{=}Matt
        \PY{k}{if} \PY{o}{[} \PY{n+nv}{\PYZdl{}user} \PY{o}{=} \PY{n+nv}{\PYZdl{}var} \PY{o}{]} \PY{p}{;} \PY{k}{then} \PY{n+nb}{echo} \PY{l+s+s2}{\PYZdq{}}\PY{l+s+s2}{string }\PY{n+nv}{\PYZdl{}user}\PY{l+s+s2}{ equals }\PY{n+nv}{\PYZdl{}var}\PY{l+s+s2}{\PYZdq{}} \PY{p}{;} \PY{k}{fi}
        \PY{k}{if} \PY{o}{[} \PY{n+nv}{\PYZdl{}user} \PY{o}{=}\PY{o}{=} \PY{n+nv}{\PYZdl{}var} \PY{o}{]} \PY{p}{;} \PY{k}{then} \PY{n+nb}{echo} \PY{l+s+s2}{\PYZdq{}}\PY{l+s+s2}{string }\PY{n+nv}{\PYZdl{}user}\PY{l+s+s2}{ equals }\PY{n+nv}{\PYZdl{}var}\PY{l+s+s2}{\PYZdq{}} \PY{p}{;} \PY{k}{fi}
        \PY{k}{if} \PY{o}{[} \PY{n+nv}{\PYZdl{}user} !\PY{o}{=} \PY{n+nv}{\PYZdl{}var} \PY{o}{]} \PY{p}{;} \PY{k}{then} \PY{n+nb}{echo} \PY{l+s+s2}{\PYZdq{}}\PY{l+s+s2}{string }\PY{n+nv}{\PYZdl{}user}\PY{l+s+s2}{ does not equal }\PY{n+nv}{\PYZdl{}var}\PY{l+s+s2}{\PYZdq{}} \PY{p}{;} \PY{k}{fi}
        \PY{k}{if} \PY{o}{[} \PYZhy{}z \PY{n+nv}{\PYZdl{}user} \PY{o}{]} \PY{p}{;} \PY{k}{then} \PY{n+nb}{echo} \PY{l+s+s2}{\PYZdq{}}\PY{l+s+s2}{string }\PY{n+nv}{\PYZdl{}user}\PY{l+s+s2}{ is empty!}\PY{l+s+s2}{\PYZdq{}}\PY{p}{;} \PY{k}{fi}
        \PY{k}{if} \PY{o}{[} \PYZhy{}n \PY{n+nv}{\PYZdl{}user} \PY{o}{]} \PY{p}{;} \PY{k}{then} \PY{n+nb}{echo} \PY{l+s+s2}{\PYZdq{}}\PY{l+s+s2}{string }\PY{n+nv}{\PYZdl{}user}\PY{l+s+s2}{ is not empty!}\PY{l+s+s2}{\PYZdq{}}\PY{p}{;} \PY{k}{fi}
        
        \PY{c+c1}{\PYZsh{}numeric                                                                                                           }
        \PY{n+nv}{a}\PY{o}{=}\PY{l+m}{1}
        \PY{n+nv}{b}\PY{o}{=}\PY{l+m}{2}
        \PY{k}{if} \PY{o}{[} \PY{n+nv}{\PYZdl{}a} \PYZhy{}eq \PY{n+nv}{\PYZdl{}b} \PY{o}{]} \PY{p}{;} \PY{k}{then} \PY{n+nb}{echo} \PY{l+s+s2}{\PYZdq{}}\PY{l+s+s2}{number }\PY{n+nv}{\PYZdl{}b}\PY{l+s+s2}{ equals }\PY{n+nv}{\PYZdl{}a}\PY{l+s+s2}{\PYZdq{}} \PY{p}{;} \PY{k}{fi}
        \PY{k}{if} \PY{o}{[} \PY{n+nv}{\PYZdl{}a} \PYZhy{}ne \PY{n+nv}{\PYZdl{}b} \PY{o}{]} \PY{p}{;} \PY{k}{then} \PY{n+nb}{echo} \PY{l+s+s2}{\PYZdq{}}\PY{l+s+s2}{number }\PY{n+nv}{\PYZdl{}b}\PY{l+s+s2}{ does not equal }\PY{n+nv}{\PYZdl{}a}\PY{l+s+s2}{\PYZdq{}} \PY{p}{;} \PY{k}{fi}
        \PY{k}{if} \PY{o}{[} \PY{n+nv}{\PYZdl{}a} \PYZhy{}gt \PY{n+nv}{\PYZdl{}b} \PY{o}{]} \PY{p}{;} \PY{k}{then} \PY{n+nb}{echo} \PY{l+s+s2}{\PYZdq{}}\PY{n+nv}{\PYZdl{}a}\PY{l+s+s2}{ is greater than }\PY{n+nv}{\PYZdl{}b}\PY{l+s+s2}{\PYZdq{}} \PY{p}{;} \PY{k}{fi}
        \PY{k}{if} \PY{o}{[} \PY{n+nv}{\PYZdl{}a} \PYZhy{}lt \PY{n+nv}{\PYZdl{}b} \PY{o}{]} \PY{p}{;} \PY{k}{then} \PY{n+nb}{echo} \PY{l+s+s2}{\PYZdq{}}\PY{n+nv}{\PYZdl{}a}\PY{l+s+s2}{ is less than }\PY{n+nv}{\PYZdl{}b}\PY{l+s+s2}{\PYZdq{}} \PY{p}{;} \PY{k}{fi}
        
        \PY{c+c1}{\PYZsh{}file/dir properties }
        \PY{n+nv}{file}\PY{o}{=}\PY{l+s+s2}{\PYZdq{}basic.txt\PYZdq{}}
        \PY{k}{if} \PY{o}{[} \PYZhy{}d \PY{n+nv}{\PYZdl{}file} \PY{o}{]} \PY{p}{;} \PY{k}{then} \PY{n+nb}{echo} \PY{l+s+s2}{\PYZdq{}}\PY{n+nv}{\PYZdl{}file}\PY{l+s+s2}{ exists and is a directory!}\PY{l+s+s2}{\PYZdq{}} \PY{p}{;} \PY{k}{fi}
        \PY{k}{if} \PY{o}{[} \PYZhy{}e \PY{n+nv}{\PYZdl{}file} \PY{o}{]} \PY{p}{;} \PY{k}{then} \PY{n+nb}{echo} \PY{l+s+s2}{\PYZdq{}}\PY{n+nv}{\PYZdl{}file}\PY{l+s+s2}{ exists!}\PY{l+s+s2}{\PYZdq{}} \PY{p}{;} \PY{k}{fi}
        \PY{k}{if} \PY{o}{[} \PYZhy{}f \PY{n+nv}{\PYZdl{}file} \PY{o}{]} \PY{p}{;} \PY{k}{then} \PY{n+nb}{echo} \PY{l+s+s2}{\PYZdq{}}\PY{n+nv}{\PYZdl{}file}\PY{l+s+s2}{ exists and is not a directory!}\PY{l+s+s2}{\PYZdq{}} \PY{p}{;} \PY{k}{fi}
        \PY{k}{if} \PY{o}{[} \PYZhy{}r \PY{n+nv}{\PYZdl{}file} \PY{o}{]} \PY{p}{;} \PY{k}{then} \PY{n+nb}{echo} \PY{l+s+s2}{\PYZdq{}}\PY{n+nv}{\PYZdl{}file}\PY{l+s+s2}{ exists and is readable!}\PY{l+s+s2}{\PYZdq{}} \PY{p}{;} \PY{k}{fi}
        \PY{k}{if} \PY{o}{[} \PYZhy{}s \PY{n+nv}{\PYZdl{}file} \PY{o}{]} \PY{p}{;} \PY{k}{then} \PY{n+nb}{echo} \PY{l+s+s2}{\PYZdq{}}\PY{n+nv}{\PYZdl{}file}\PY{l+s+s2}{ exists and has size greater than zero!}\PY{l+s+s2}{\PYZdq{}} \PY{p}{;} \PY{k}{fi}
        \PY{k}{if} \PY{o}{[} \PYZhy{}w \PY{n+nv}{\PYZdl{}file} \PY{o}{]} \PY{p}{;} \PY{k}{then} \PY{n+nb}{echo} \PY{l+s+s2}{\PYZdq{}}\PY{n+nv}{\PYZdl{}file}\PY{l+s+s2}{ exists and is writable!}\PY{l+s+s2}{\PYZdq{}} \PY{p}{;} \PY{k}{fi}
        \PY{k}{if} \PY{o}{[} \PYZhy{}x \PY{n+nv}{\PYZdl{}file} \PY{o}{]} \PY{p}{;} \PY{k}{then} \PY{n+nb}{echo} \PY{l+s+s2}{\PYZdq{}}\PY{n+nv}{\PYZdl{}file}\PY{l+s+s2}{ exists and is executable!}\PY{l+s+s2}{\PYZdq{}} \PY{p}{;} \PY{k}{fi}
        
        \PY{c+c1}{\PYZsh{} NOTE: Since everything\PYZsq{}s on the same line, I need ;s between                                                     }
        \PY{c+c1}{\PYZsh{} the if and the then and between the then and the fi.}
\end{Verbatim}


    \begin{Verbatim}[commandchars=\\\{\}]
string Matt does not equal adam
string Matt is not empty!
number 2 does not equal 1
1 is less than 2
basic.txt exists!
basic.txt exists and is not a directory!
basic.txt exists and is readable!
basic.txt exists and has size greater than zero!
basic.txt exists and is writable!
basic.txt exists and is executable!

    \end{Verbatim}

    Additionally, you can use the standard set of conditional operators:

    \begin{Shaded}
\begin{Highlighting}[]
\KeywordTok{if [} \OtherTok{$condition1}\KeywordTok{ ]} \KeywordTok{|| [} \OtherTok{$condition2}\KeywordTok{ ]} \NormalTok{; }\KeywordTok{do} \KeywordTok{echo} \StringTok{"either condition1 or condition2 are true"}\KeywordTok{;} \KeywordTok{done}
\KeywordTok{if [} \OtherTok{$condition1}\KeywordTok{ ]} \KeywordTok{&& [} \OtherTok{$condition2}\KeywordTok{ ]} \NormalTok{; }\KeywordTok{do} \KeywordTok{echo} \StringTok{"condition1 and condition2 are true"}\KeywordTok{;} \KeywordTok{done}
\KeywordTok{if [} \OtherTok{!} \OtherTok{$condition1}\KeywordTok{ ]} \NormalTok{; }\KeywordTok{do} \KeywordTok{echo} \StringTok{"condtion1 is not true"}\KeywordTok{;} \KeywordTok{done}
\end{Highlighting}
\end{Shaded}

    \subsubsection{C. For Loops}\label{c.-for-loops}

A for loop has a similar structure as in R and for loops iterate for
each item provided:

    \begin{Shaded}
\begin{Highlighting}[]
\KeywordTok{for} \KeywordTok{var} \NormalTok{in str1 str2 .. strn}
\KeywordTok{do}
   \CommentTok{# commands}
   \CommentTok{# $var is available for references, set to str1 -> strn in each loop}
\KeywordTok{done}
\end{Highlighting}
\end{Shaded}

    \begin{Verbatim}[commandchars=\\\{\}]
{\color{incolor}In [{\color{incolor}64}]:} \PYZpc{}\PYZpc{}bash
         
         \PY{k}{for} i in \PY{l+m}{1} \PY{l+m}{2} \PY{l+m}{3} \PY{l+m}{4} a b c
         \PY{k}{do}
             \PY{n+nb}{echo} \PY{l+s+s2}{\PYZdq{}}\PY{l+s+s2}{Item: }\PY{n+nv}{\PYZdl{}i}\PY{l+s+s2}{\PYZdq{}}
         \PY{k}{done}
\end{Verbatim}


    \begin{Verbatim}[commandchars=\\\{\}]
Item: 1
Item: 2
Item: 3
Item: 4
Item: a
Item: b
Item: c

    \end{Verbatim}

    \subsection{6. Output to stdout vs.
stderr}\label{output-to-stdout-vs.-stderr}

A bash script is just like other command line tools we've seen so far
with respect to output and input. The same standard file names and file
descriptor numbers are automatically provided as well as the ability to
redirect them.

When you put a command in a script that writes to standard out, like say
head or cat command, the output will also be part of the standard output
of the script, just like echo was above. Similarly, if a command has an
error and writes to standard error, that will also be a part of the
script's standard error.

If you want to write to standard error directly in your script, you will
redirect the output of the echo command.

    \begin{Verbatim}[commandchars=\\\{\}]
{\color{incolor}In [{\color{incolor}65}]:} \PYZpc{}\PYZpc{}bash
         
         \PY{n+nb}{echo} \PY{l+s+s2}{\PYZdq{}ERROR: Something bad happened\PYZdq{}}        \PY{c+c1}{\PYZsh{}\PYZlt{}\PYZhy{}\PYZhy{} direct to stdout}
         
         \PY{n+nb}{echo} \PY{l+s+s2}{\PYZdq{}ERROR: Something bad happened\PYZdq{}}  \PY{l+m}{1}\PYZgt{}\PY{p}{\PYZam{}}\PY{l+m}{2}  \PY{c+c1}{\PYZsh{}\PYZlt{}\PYZhy{}\PYZhy{} redirect file descriptor 1 to 2}
\end{Verbatim}


    \begin{Verbatim}[commandchars=\\\{\}]
ERROR: Something bad happened

    \end{Verbatim}

    \begin{Verbatim}[commandchars=\\\{\}]
ERROR: Something bad happened

    \end{Verbatim}

    \begin{Verbatim}[commandchars=\\\{\}]
{\color{incolor}In [{\color{incolor}66}]:} \PYZpc{}\PYZpc{}bash
         
         \PY{n+nv}{H}\PY{o}{=}\PY{l+m}{8}
         
         \PY{k}{if} \PY{o}{[} \PY{n+nv}{\PYZdl{}H} \PY{o}{=}\PY{o}{=} \PY{l+m}{10} \PY{o}{]}
         \PY{k}{then}
             \PY{n+nb}{echo} \PY{l+s+s2}{\PYZdq{}The equality is TRUE\PYZdq{}}
         \PY{k}{else}
             \PY{n+nb}{echo} \PY{l+s+s2}{\PYZdq{}ERROR: The equality is FALSE\PYZdq{}} \PY{l+m}{1}\PYZgt{}\PY{p}{\PYZam{}}\PY{l+m}{2}
         \PY{k}{fi}
\end{Verbatim}


    \begin{Verbatim}[commandchars=\\\{\}]
ERROR: The equality is FALSE

    \end{Verbatim}

    \subsection{7. There's more...}\label{theres-more...}

\begin{center}\rule{0.5\linewidth}{\linethickness}\end{center}

We only scratched the surface of the Unix shell in this exercise. There
are many other commands that can be combined in an infinite number of
ways. Many repetitive tasks that would take hours of manual work can be
done in a few minutes by writing the appropriate commands. Mastering the
Unix shell may take a lot of effort, but it leads to dramatic time gains
in the long term.

Here are a few references:

\begin{itemize}
\tightlist
\item
  Linux tutorial at https://ryanstutorials.net/linuxtutorial/
\item
  Bash commands at https://ss64.com/bash/
\item
  Learn Bash in Y minutes, at https://learnxinyminutes.com/docs/bash/
\item
  Learn the shell interactively, at http://www.learnshell.org/
\item
  The fish shell, at https://fishshell.com/
\item
  xonsh, a Python-powered shell, at http://xon.sh/
\item
  Windows Subsystem for Linux, at
  https://docs.microsoft.com/windows/wsl/about
\end{itemize}

    \begin{center}\rule{0.5\linewidth}{\linethickness}\end{center}

This tutorial was adapted from
https://ipython-books.github.io/21-learning-the-basics-of-the-unix-shell/
***


    % Add a bibliography block to the postdoc
    
    
    
    \end{document}
